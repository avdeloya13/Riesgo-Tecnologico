\documentclass[12pt]{article}
\usepackage{enumitem}
\usepackage[utf8]{inputenc}
\usepackage[T1]{fontenc}
\usepackage{fancyhdr}
\usepackage[spanish,mexico,es-lcroman]{babel} % If you write in French
\usepackage{xcolor,graphicx}
\usepackage[top=0.6in,bottom=0.6in,right=0.7in,left=0.7in]{geometry}
%------ Paquete para ajustar imágen --------%
\usepackage{float}
%%%%%%%%%%%%%%%%%%%%%%%%%%%%%%%%%%%%%%%%%
%colors
\definecolor{applegreen}{rgb}{0.55, 0.71, 0.0}

% Set up fancy header/footer
\pagestyle{fancy}
\fancyhead[LO,L]{UNAM}
%\fancyhead[CO,C]{ECE305 - Homework Example}
\fancyfoot[LO,L]{Facultad De Ciencias, 2025-I}
\fancyfoot[CO,C]{\thepage}
\fancyfoot[RO,R]{\tt CyberLink}
\renewcommand{\headrulewidth}{0.4pt}
\renewcommand{\footrulewidth}{0.4pt}

\begin{document}
\begin{titlepage}
% \pagecolor{blue!10}
\begin{center}

\textsc{\Large \textcolor{black}{\textbf{Universidad Nacional Autónoma De México}}}\\[0.4cm]	

{\huge \Large \uppercase{Facultad de Ciencias} \\[0.4cm] }
{\huge \Large \uppercase{Riesgo Tecnológico} \\[1cm] }

\includegraphics[scale=0.45]{IMAGE/Logo.png}\\[1cm]

%\textsc{\Large \textbf{XXXXXXXXXXXXX:}}\\[0.4cm]	

%%% Título
\rule{\linewidth}{0.7mm} \\[0.4cm]
\includegraphics[scale=0.45]{IMAGE/Empresa.png}\\[0.1cm]
\rule{\linewidth}{0.7mm} \\[1cm]


\color{black!80!black}{\large \textbf{Profesor:}}\\[0.4cm]

\begin{tabular}{l}
\large Selene Marisol Martínez Ramírez \\[0.5cm]
\end{tabular}

\color{black!80!black}{\large \textbf{INTEGRANTES:}}\\[0.6cm]
\color{black}
\centering
\begin{tabular}{l}


\large Deloya Andrade Ana Valeria \\[0.4cm]

\large   Cortés Jiménez Carlos Daniel \\[0.4cm]

\large  Cruz Gonzalez Irvin Javier \\[0.4cm]

\large  Castro Reyes Angel  \\[0.4cm]

\large   Cruz Blanco Gabriela \\[0.4cm]
\end{tabular}

\vfill

\end{center}
\end{titlepage}

\section*{Analísis FODA: CatsDics}

\noindent CatDics es una tienda digital dirigida a los audiófilos y aquellos coleccionistas que buscan material musical de cualquier época, ofrecemos a nuestros clientes todo tipo de contenido musical, tal como CD's, vinilos o incluso cassettes. 
Permitimos a los usuarios explorar cada categoría, como el medio en el que se encuentra el contenido, géneros musicales y algunos otros detalles acerca de nuestros productos, además si no contamos con el producto coleccionable que desea adquirir, puede solicitarlo y lo obtendremos de manera eficaz y rápida.
\\

\begin{itemize}
	\item \textbf{Fortalezas F} 

	\begin{itemize}
		\item \textbf{Búsqueda personalizada para los clientes:} El sistema tiene la capacidad de localizar coleccionables o música que usualmente es difícil de encontrar en una tienda de música común.

		\item \textbf{Gran amplitud de productos en el catalogo:} Se pueden encontrade desde vinilos antiguos hasta cassettes y CDs actuales, cubriendo múltiples épocas y formatos para todo tipo de clientes.

		\item \textbf{Precios accesibles:} Ofrecer precios accesibles y promociones atractivas hace que el servicio sea más atractivo para los clientes.

		\item \textbf{Plataforma intuitiva:} Fácil uso de la plataforma dividida por categorías, géneros y detalles, mejorando y facilitando la experiencia del usuario.

		\item \textbf{Estrecha relación con el cliente:} Seguimiento constante del pedido para generar confianza.

	\end{itemize}

	\item \textbf{Oportunidades O}

	\begin{itemize}
		\item \textbf{Incremento del comercio electrónico en el campo de la música:} Aprovechar la tendencia creciente de compras en línea, así como hacer competencia a compañías de música como spotify para tener precios más justos.

		\item \textbf{Mercado global:} Expandirse la empresa de ser nacional a nivel internacional para captar un público más amplio.

		\item \textbf{Colaboraciones con artistas y sellos discográficos:} Buscar alianzas estratégicas para tener más representación y visibilidad.

		\item \textbf{Personalización del servicio:} Implementar algoritmos de inteligencia artificial para ralizar recomendaciones basadas en gustos de los cliente.

		\item \textbf{Redes sociales y marketing digital:} Promocionar la empresa a través de plataformas populares para llegar a más personas.

	\end{itemize}

	\item \textbf{Debilidades D}

	\begin{itemize}
		\item \textbf{Dependencia del inventario:} Mantener un stock suficiente y actualizado dada un demanda alta de producto puede llegar a ser complicado y costoso.
		
		\item \textbf{Gastos en logística:} Costos asociados al transporte y adquisición de productos específicos para los usuarios puede incrementar los costos.
		
		\item \textbf{Falta de reconocimiento inicial:} Al ser una empresa nueva, se requiere construir confianza y posicionamiento en el mercado ante la competencia.
		
		\item \textbf{Tiempo de entrega:} La obtención de productos personalizados puede retrasarse, afectando la satisfacción del cliente.
		
		\item \textbf{Competencia de plataformas establecidas:} Competir con gigantes del comercio electrónico será un desafío tales como spotify.

	\end{itemize}

	\item \textbf{Amenazas A}

	\begin{itemize}
		\item \textbf{Competencia intensa:} Tiendas físicas y plataformas en línea ya consolidadas, como Amazon, pueden llegar a ser un problema.
		
		\item \textbf{Fluctuación en precios de coleccionables:} Los precios de productos raros pueden ser inestables y afectar los márgenes de ganancia establecido.
		
		\item \textbf{Cambios en hábitos de consumo por los clientes:} La transición hacia servicios de streaming puede reducir el interés en formatos físicos que es lo que la tendencia indica.
		
		\item \textbf{Regulaciones internacionales:} Las leyes de importación y exportación pueden complicar la adquisición de productos y retrasar los pedidos.
		
		\item \textbf{Crisis económica:} Una disminución en el poder adquisitivo del país puede impactar las ventas de artículos no esenciales.

	\end{itemize}

\end{itemize}


 \section*{FODA}

\begin{itemize}
	\item \textbf{Fortalezas} 

	\begin{itemize}
		\item Búsqueda personalizada para los clientes:

		\item Gran amplitud de productos en el catalogo

		\item Precios accesibles:

		\item Plataforma intuitiva

		\item Estrecha relación con el cliente:

	\end{itemize}

	\item \textbf{Oportunidades O}

	\begin{itemize}
		\item Incremento del comercio electrónico en el campo de la música:

		\item Mercado global

		\item Colaboraciones con artistas y sellos discográficos

		\item Personalización del servicio

		\item Redes sociales y marketing digital
	\end{itemize}

	\item \textbf{Debilidades D}

	\begin{itemize}
		\item Dependencia del inventario
		
		\item Gastos en logística
		
		\item Falta de reconocimiento inicial
		
		\item Tiempo de entrega
		
		\item Competencia de plataformas establecidas

	\end{itemize}

	\item \textbf{Amenazas A}

	\begin{itemize}
		\item Competencia intensa
		
		\item Fluctuación en precios de coleccionables
		
		\item Cambios en hábitos de consumo por los clientes
		
		\item Regulaciones internacionales
		
		\item Crisis económica

	\end{itemize}

\end{itemize}

\section*{Estrategías FODA}

\begin{itemize}
	\item \textbf{Estrategias FO (Fortalezas + Oportunidades):} 

	\begin{itemize}
		\item \textbf{Impulsar promociones por redes sociales:} Utilizar las promociones y la personalización del servicio para captar nuevos clientes a través de campañas publicitarias en plataformas digitales, como instagram, tik tok, etc
		
		\item \textbf{Expandir la empresa internacionalmente:} Aprovechar la capacidad de búsqueda personalizada para llegar a coleccionistas globales, potenciando la visibilidad de la empresa.
		
		\item \textbf{Colaboraciones estratégicas:} Tener exclusividad de productos asociandonos con artistas, sellos discográficos.

	\end{itemize}

	\item \textbf{Estrategias DO (Debilidades + Oportunidades):}

	\begin{itemize}
		\item \textbf{Optimizar tiempos de entrega:} Mejorar la logística mediante acuerdos con proveedores confiables para llegar a un acuerdo comercial y para garantizar entregas más rápidas para los cliente.
		
		\item \textbf{Incrementar la confianza del cliente:} Realizar reseñas y casos de éxito en la plataforma para mitigar la falta de reconocimiento inicial.

	\end{itemize}

	\item \textbf{Estrategias FA (Fortalezas + Amenazas):}

	\begin{itemize}
		\item \textbf{Diferenciación frente a la competencia:} Hacernos destacar ante la competencia por medio de la búsqueda personalizada y precios mas accesibles ante los competidores.
		
		\item \textbf{Adaptación a cambios de mercado:} Ampliar la oferta de productos digitales complementarios, como lo es el mercado de videojuegos, tener ediciones especiales de articulos que sean atractivos para los clientes y asi mitigar a la amenaza del streaming.

	\end{itemize}

	\item \textbf{Estrategias DA (Debilidades + Amenazas):}

	\begin{itemize}
		\item \textbf{Plan de contingencia en logística:} Diseñar un sistema para manejar retrasos en la obtención de productos raros, comunicando claramente con el cliente y ofreciendo alternativas.
		
		\item \textbf{Análisis constante del mercado:} Monitorear tendencias y regulaciones para ajustar el modelo de negocio y evitar impactos por cambios económicos o legales.

	\end{itemize}

\end{itemize}

\section*{FO-DO-FA-DA}

\begin{itemize}
	\item \textbf{Fortalezas + Oportunidades: FO} 

	\begin{itemize}
		\item Impulsar las redes sociales
		
		\item Expandir el alcance internacional
		
		\item Colaboraciones estratégicas

	\end{itemize}

	\item \textbf{Debilidades + Oportunidades: DO}

	\begin{itemize}
		\item Optimizar tiempos de entrega
		
		\item Incrementar la confianza del cliente
		
		\item Diversificar las estrategias de marketing

	\end{itemize}

	\item \textbf{Fortalezas + Amenazas: FA}

	\begin{itemize}
		\item Diferenciación frente a la competencia
		
		\item Adaptación a cambios de mercado

	\end{itemize}

	\item \textbf{Debilidades + Amenazas: DA}

	\begin{itemize}
		\item Plan de contingencia en logística
		
		\item Análisis constante del mercado

	\end{itemize}

\end{itemize}

`\end{document}

