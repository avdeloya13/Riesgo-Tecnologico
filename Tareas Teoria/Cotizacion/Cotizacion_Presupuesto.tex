\documentclass[12pt]{article}
\usepackage{enumitem}
\usepackage[utf8]{inputenc}
\usepackage[T1]{fontenc}
\usepackage{fancyhdr}
\usepackage[spanish,mexico,es-lcroman]{babel} % If you write in French
\usepackage{xcolor,graphicx}
\usepackage[top=0.6in,bottom=0.6in,right=0.7in,left=0.7in]{geometry}
%------ Paquete para ajustar imágen --------%
\usepackage{float}
%%%%%%%%%%%%%%%%%%%%%%%%%%%%%%%%%%%%%%%%%
%colors
\definecolor{applegreen}{rgb}{0.55, 0.71, 0.0}

% Set up fancy header/footer
\pagestyle{fancy}
\fancyhead[LO,L]{UNAM}
%\fancyhead[CO,C]{ECE305 - Homework Example}
\fancyhead[RO,R]{\textit{Tarea teoría}}
\fancyfoot[LO,L]{Facultad De Ciencias, 2025-I}
\fancyfoot[CO,C]{\thepage}
\fancyfoot[RO,R]{Riesgo Tecnológico}
\renewcommand{\headrulewidth}{0.4pt}
\renewcommand{\footrulewidth}{0.4pt}

\begin{document}
\begin{titlepage}
% \pagecolor{blue!10}
\begin{center}

\textsc{\Large \textcolor{black}{\textbf{Universidad Nacional Autónoma De México}}}\\[0.4cm]	

{\huge \Large \uppercase{Facultad de Ciencias} \\[0.4cm] }
{\huge \Large \uppercase{Riesgo Tecnológico} \\[1cm] }

\includegraphics[scale=0.45]{images/TR.png}\\[1cm]

%\textsc{\Large \textbf{XXXXXXXXXXXXX:}}\\[0.4cm]	

%%% Título
\rule{\linewidth}{0.7mm} \\[0.4cm]
{ \huge \bfseries\color{cyan!70!black} Tarea de Teoría \\[0.3cm] }
{ \huge \bfseries\color{black} Cotización y Presupuesto en Proyectos \\[0.3cm] }
\rule{\linewidth}{0.7mm} \\[1cm]


\color{black!80!black}{\large \textbf{Profesor:}}\\[0.4cm]

\begin{tabular}{l}
\large Selene Marisol Martínez Ramírez \\[0.5cm]
\end{tabular}

\color{black!80!black}{\large \textbf{INTEGRANTES:}}\\[0.6cm]
\color{black}
\centering
\begin{tabular}{l}


\large Deloya Andrade Ana Valeria \\[0.4cm]

\large   Cortés Jiménez Carlos Daniel \\[0.4cm]

\large  Cruz Gonzalez Irvin Javier \\[0.4cm]

\large  Castro Reyes Angel  \\[0.4cm]

\large   Cruz Blanco Gabriela \\[0.4cm]
\end{tabular}

\vfill

\end{center}
\end{titlepage}

 \section*{Paso 1: Definición del Proyecto}

 \noindent Risk es un juego de mesa de estrategia con el objetivo de conquistar territorios, cada jugador escoge un color que serán representados como líderes políticos cuyo propósito será intentar proteger su territorio y su vez expandirlo conquistando otros países. Aquellos jugadores con más control territorial ganan. \\

 \noindent Para este proyecto desarrollaremos una versión física del juego Risk, con la visión de crear en un futuro versión digital del mismo. Nuestro objetivo es generar una experiencia única para los compradores, en la que puedan crear recuerdos que no olvidarán con amigos y familiares, teniendo así un diseño único, con los materiales de la más alta calidad, no solo ofrecemos un juego sino una experiencia de socialización para que jugadores generen un vínculo entre ellos.

\begin{enumerate}
	\item \textbf{Alcance del Proyecto:} 

	\begin{itemize}
		\item Expandir la version clásica de Risk para hacerla más emocionante.

		\item Una versión más imersiva para el usuario.

		\item Reglas más realistas para que los jugadores puedan experimentar como fueron las guerras.

		\item Tablero, fichas nuevas y cartas de juego teniendo en cuenta desarrollo sustentable.

		\item Ganancias del 17\%

	\end{itemize}

	\item \textbf{Objetivos del Proyecto:}

	\begin{itemize}
		\item Crear una nueva versión mas realista e imersiva del tablero de Risk, junto con nuevos elementos.

		\item Realizar los componentes del juego con un alta calidad y atención al diseño.
		.
		\item Reestructurar el manual de reglas así como añadir nuevas reglas.

		\item Utilizar materiales de alta calidad.

		\item Cumplir con los tiempos esperado para la producción en masa del tablero y maximizando el presupuesto estimado sin sacrificar la calidad.

	\end{itemize}

\end{enumerate}


 \section*{Paso 2: Identificar Recursos y Costos}


A continuación, señalamos los recursos necesarios para llevar a cabo el proyecto, así como los costos de producción.

\begin{enumerate}
	\item Personal

	\begin{itemize}
		\item Diseñador gráfico: 1 

		Costo por persona: \$20,000

		\item Desarrollador de reglas/juego: 1

		Costo por persona: \$5,000

		\item Gerente del proyecto: 1

		Costo por persona: \$30,000

		\item Equipo de supervisión de ensamblaje y producción:  3

		Costo por persona: \$18,000

		\item Ingeniero en electrónica: 2

		Costo por persona: \$15,000

		\item Testers (Playtesting): 5

		Costo por persona: \$3,000

		\item Consultoría: 1

		Costo por persona: \$30,000

	\end{itemize}

	\item Materiales 

	\begin{itemize}
		\item Tablero de juego

		Costo por unidad: \$295

		\item Fichas

		Costo por unidad: \$230

		\item Cartas

		Costo por unidad: \$170

		\item Dados

		Costo por unidad: \$120

		\item Cajas de empaque

		Costo por unidad: \$200
	\end{itemize}

	\item Equipos

	\begin{itemize}
		\item Impresora 3D

		Costo por unidad: \$8,000

		\item Licencia software de diseño gráfico

		Costo mensual: \$1,000

		\item Computadoras

		Costo por unidad: \$7,000

	\end{itemize}

	\item Servicios

	\begin{itemize}
		\item Fabricación de tableros, fichas y cartas

		Costo por unidad de set de juego: \$500

		\item Envío y distribución

		Costo por unidad de envio: \$400

		\item Publicidad y marketing

		Costo: \$ 4,500

		\item Mantenimiento

		Costo mensual de equipos: \$ 6,000
	\end{itemize}

\end{enumerate}


 \section*{Paso 3: Calcular los Costos}

 	\noindent En base a los recursos necesarios para llevar a cabo el proyecto, se establecen los costos en función de la duración y cantidad. Considerando que algunas tareas se realizan de forma simultánea, la duración total sería de aproximadamente 2 meses.

    \begin{enumerate}
	\item Personal

	\begin{itemize}
		\item Diseñador gráfico: 1 

        Tiempo estimado de 100 horas, añadiendo un 10\% extra para cambios de último momento o ajustes.
        \begin{itemize}
            \item 110 horas $\times$ 200/hora = \textbf{22000}
            \item Costo por persona: \$22,000
        \end{itemize}	

     
		\item Desarrollador de reglas/juego: 1

           50 horas, pretendiendo agregar un 15\% más para la exposición de las mismas, revisiones y aclaraciones.
            \begin{itemize}
            \item 57.5 horas $\times$ 100/hora = \textbf{5750}
            \item Costo por persona: \$5750
            \end{itemize}	

		\item Gerente del proyecto: 1

            100 horas para revisiones y aclaraciones.
            \begin{itemize}
            \item 100 horas $\times$ 300/hora = \textbf{30,000}
            \item Costo por persona: \$3000
            \end{itemize}	

		\item Equipo de supervisión de ensamblaje y producción:  3

          200 horas para supervisión, añadiendo un 10\% extra para ajustes.
           100 horas para revisiones y aclaraciones.
            \begin{itemize}
            \item 220 horas $\times$ 85/hora = \textbf{18,700}
            \item Costo por persona: \$18,700
            \end{itemize}	

		Costo por persona: \$18,700

		\item Ingeniero en electrónica: 2
  
          200 horas para supervisión, añadiendo un 10\% extra para ajustes.
           100 horas para revisiones y aclaraciones.
            \begin{itemize}
            \item 220 horas $\times$ 90/hora = \textbf{19,800}
            \item Costo por persona: \$19,800
            \end{itemize}

		\item Testers(Playtesting): 5
	
         30 horas donde se aplica las pruebas para asegurar la calidad del producto. Añadimos un 10\% adicional de tiempo para pruebas de jugabilidad, accesibilidad y usabilidad.
          \begin{itemize}
            \item 30 horas $\times$ 100/hora = \textbf{3,300}
            \item Costo por persona: \$3,300
            \end{itemize}

		\item Consultoría: 1

          100 horas para el desarrollo de concepto y análisis de mercado.
            \begin{itemize}
            \item 110 horas $\times$ 300/hora = \textbf{33,000}
            \item Costo por persona: \$33,000
            \end{itemize}

	\end{itemize}

	\newpage

	\item Materiales 

	\begin{itemize}
		\item Tablero de juego
        \begin{itemize}
            \item 8 horas $\times$ 40/hora = \textbf{320}
            \item Costo por unidad: \$320
        \end{itemize}

		\item Fichas
        \begin{itemize}
            \item 8 horas $\times$ 40/hora = \textbf{320}
            \item Costo por unidad \$320
        \end{itemize}

		\item Cartas
        \begin{itemize}
            \item 8 horas $\times$ 30/hora = \textbf{320}
            \item Costo por unidad: \$240
        \end{itemize}

		\item Dados
        \begin{itemize}
            \item 8 horas $\times$ 20/hora = \textbf{160}
            \item Costo por unidad: \$160
        \end{itemize}

		\item Cajas de empaque
        \begin{itemize}
            \item 8 horas $\times$ 30/hora = \textbf{240}
            \item Costo por unidad: \$240
        \end{itemize}

	\end{itemize}

	\item Equipos

	\begin{itemize}
		\item Impresora 3D
		\begin{itemize}
            \item 12 horas $\times$ 100/hora = \textbf{12000} 
            \item Costo por equipo: \$12,000
        \end{itemize}

		Costo por unidad: \$8,000

		\item Licencia software de diseño gráfico
		\begin{itemize}
            \item 2 meses $\times$ \$1,000/mes = \textbf{2000}
            \item Costo por equipo: \$2000
        \end{itemize}

		Costo mensual: \$1,000

		\item Computadoras

		Se utilizarán 2 computadoras, una para el diseñador gráfico y otra para el ingeniero electrónico.
		\begin{itemize}
            \item 2 computadoras $\times$ 240 horas $\times$ \$50/hora = \textbf{24,000}
            \item Costo por equipo: \$ 24,000
        \end{itemize}

	\end{itemize}

	

	\item Servicios

	\begin{itemize}

		\item Fabricación de tableros, fichas y cartas

		Producción estimada durante el desarrollo del proyecto.
		\begin{itemize}
            \item 300 unidades $\times$ \$500 por set = \textbf{150,000}
			\item Costo por servicio: \$150,000
        \end{itemize}

		\item Envío y distribución
		
		Al finalizar el proyecto.
		\begin{itemize}
            \item 300 envíos $\times$ \$400 por envío = \textbf{120,000}
            \item Costo por servicio: \$120,000
        \end{itemize}

		\item Publicidad y marketing

		\begin{itemize}
			\item \$4,500/2 meses = \textbf{4,500}
            \item Costo por servicio: \$4,500
        \end{itemize}

		\item Mantenimiento

		Costo de mantenimiento por mes.
		\begin{itemize}
            \item 2 meses $\times$ \$6,000/mes = \textbf{12,000}
            \item Costo por servicio: \$12,000
        \end{itemize}

	\end{itemize}

\end{enumerate}
    


 \section*{Paso 4: Crear un Presupuesto Detallado}
    

`\end{document}

