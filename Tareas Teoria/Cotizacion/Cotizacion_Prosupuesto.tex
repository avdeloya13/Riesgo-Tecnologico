\documentclass[12pt]{article}
\usepackage{enumitem}
\usepackage[utf8]{inputenc}
\usepackage[T1]{fontenc}
\usepackage{fancyhdr}
\usepackage[spanish,mexico,es-lcroman]{babel} % If you write in French
\usepackage{xcolor,graphicx}
\usepackage[top=0.6in,bottom=0.6in,right=0.7in,left=0.7in]{geometry}
%------ Paquete para ajustar imágen --------%
\usepackage{float}
%%%%%%%%%%%%%%%%%%%%%%%%%%%%%%%%%%%%%%%%%
%colors
\definecolor{applegreen}{rgb}{0.55, 0.71, 0.0}

% Set up fancy header/footer
\pagestyle{fancy}
\fancyhead[LO,L]{UNAM}
%\fancyhead[CO,C]{ECE305 - Homework Example}
\fancyhead[RO,R]{\textit{Tarea teoria}}
\fancyfoot[LO,L]{Facultad De Ciencias, 2025-I}
\fancyfoot[CO,C]{\thepage}
\fancyfoot[RO,R]{Riesgo Tecnológico}
\renewcommand{\headrulewidth}{0.4pt}
\renewcommand{\footrulewidth}{0.4pt}

\begin{document}
\begin{titlepage}
% \pagecolor{blue!10}
\begin{center}

\textsc{\Large \textcolor{black}{\textbf{Universidad Nacional Autónoma De México}}}\\[0.4cm]	

{\huge \Large \uppercase{Facultad de Ciencias} \\[0.4cm] }
{\huge \Large \uppercase{Riesgo Tecnológico} \\[1cm] }

\includegraphics[scale=0.45]{images/TR.png}\\[1cm]

%\textsc{\Large \textbf{XXXXXXXXXXXXX:}}\\[0.4cm]	

%%% Título
\rule{\linewidth}{0.7mm} \\[0.4cm]
{ \huge \bfseries\color{cyan!70!black} Tarea de Teoría \\[0.3cm] }
{ \huge \bfseries\color{black} Análisis de un Sistema de Información \\[0.3cm] }
\rule{\linewidth}{0.7mm} \\[1cm]


\color{black!80!black}{\large \textbf{Profesor:}}\\[0.4cm]

\begin{tabular}{l}
\large Selene Marisol Martínez Ramírez \\[0.5cm]
\end{tabular}

\color{black!80!black}{\large \textbf{INTEGRANTES:}}\\[0.6cm]
\color{black}
\centering
\begin{tabular}{l}


\large Deloya Andrade Ana Valeria \\[0.4cm]

\large   Cortés Jiménez Carlos Daniel \\[0.4cm]

\large  Cruz Gonzalez Irvin Javier \\[0.4cm]

\large  Castro Reyes Angel  \\[0.4cm]

\large   Cruz Blanco Gabriela \\[0.4cm]
\end{tabular}

\vfill

\end{center}
\end{titlepage}

 \section*{Paso 1: Definición del Proyecto}

  Risk es un juego de mesa de estrategia con el objetivo de conquistar territorios. Cada jugador escoge un color que seran representados como lideres politicos que trataran de proteger su territorio y su vez expandir su territorio conquistando otros paises los jugadores que tengan mas control territotial ganan.

\begin{enumerate}
	\item \textbf{Alcance del Proyecto:} 

	\begin{itemize}
		\item Expandir la version clásica de Risk para hacerla más emocionante

		\item Una versión más imersiva para el usuario

		\item Reglas más realistas para que los jugadores puedan experimentar como fueron las guerras
		.
		\item Tablero, fichas nuevas y cartas de juego teniendo en cuenta desarrollo sustentable

		\item Ganancias del 17\%

	\end{itemize}

	\item \textbf{Objetivos del Proyecto:}

	\begin{itemize}
		\item Crear una nueva versión mas realista e imersiva del tablero de Risk, junto con nuevos elementos.

		\item Realizar los componentes del juego con un alta calidad y atención al diseño
		.
		\item Reestructurar el manual de reglas así como añadir nuevas reglas.

		\item Cumplir con los tiempos esperado para la producción en masa del tablero y máximizando el presupuesto estimado sin sacrificar la calidad.

	\end{itemize}

\end{enumerate}


 \section*{Paso 2: Identificar Recursos y Costos}


A continuación, señalamos los recursos necesarios para llevar a cabo el proyecto, así como los costos de proudcción.

\begin{enumerate}
	\item Personal

	\begin{itemize}
		\item Diseñador gráfico: 1 

		Costo por persona: \$20,000

		\item Desarrollador de reglas/juego: 1

		Costo por persona: \$5,000

		\item Gerente del proyecto: 1

		Costo por persona: \$30,000

		\item Equipo de supervisión de ensamblaje y producción:  3

		Costo por persona: \$18,000

		\item Ingeniero en electronica: 2

		Costo por persona: \$15,000

		\item Testers: 5

		Costo por persona: \$3,000

		\item Consultoría: 1

		Costo por persona: \$30,000

	\end{itemize}

	\item Materiales 

	\begin{itemize}
		\item tablero de juego

		Costo por unidad: \$295

		\item Fichas

		Costo por unidad: \$230

		\item Cartas

		Costo por unidad: \$170

		\item Dados

		Costo por unidad: \$120

		\item Cajas de empaque

		Costo por unidad: \$200
	\end{itemize}

	\item Equipos

	\begin{itemize}
		\item Impresora 3D

		Costo por unidad: \$8,000

		\item Licencia software de diseño gráfico

		Costo mensual: \$1,000

		\item Computadoras

		Costo por unidad: \$7,000

	\end{itemize}

	\item Servicios

	\begin{itemize}
		\item Fabricación de tableros, fichas y cartas

		Costo por unidad de set de juego: \$500

		\item Envío y distribución

		Costo por unidad de envio: \$400

		\item Publicidad y marketing

		Costo: \$ 4,500

		\item Mantenimiento

		Costo mensual de equipos: \$ 6,000
	\end{itemize}





\end{enumerate}



 \section*{Paso 3: Calcular los Costos}


 \section*{Paso 4: Crear un Presupuesto Detallado}
    

`\end{document}

